\documentclass{article}

\PassOptionsToPackage{numbers, compress}{natbib}

\usepackage[final]{nips_2016}
\usepackage[utf8]{inputenc} % allow utf-8 input
\usepackage[T1]{fontenc}    % use 8-bit T1 fonts
\usepackage{hyperref}       % hyperlinks
\usepackage{url}            % simple URL typesetting
\usepackage{booktabs}       % professional-quality tables
\usepackage{amsfonts}       % blackboard math symbols
\usepackage{nicefrac}       % compact symbols for 1/2, etc.
\usepackage{microtype}      % microtypography
\usepackage{amssymb}
\usepackage{amsmath}
\usepackage{amsthm}
\usepackage{caption}
\usepackage{subcaption}
\usepackage{graphicx}
\usepackage{color}
\usepackage{algorithm}
\usepackage{algpseudocode}

\newcommand{\glnote}[1]{\textcolor{red}{[GL: #1]}}

\title{Adversarial Training against Systematic Uncertainty}

\author{
  Gilles Louppe \\
  New York University\\
  \texttt{g.louppe@nyu.edu} \\
  \And
  Michael Kagan\\
  SLAC National Accelerator Laboratory\\
  \texttt{makagan@slac.stanford.edu}\\
  \And
  Kyle Cranmer \\
  New York University\\
  \texttt{kyle.cranmer@nyu.edu}
}

\theoremstyle{plain}
\newtheorem{theorem}{Theorem}
\newtheorem{proposition}[theorem]{Proposition}

\begin{document}

\maketitle

% ==============================================================================

\begin{abstract}

Lorem ipsum dolor sit amet, consectetur adipiscing elit...

\end{abstract}


% ==============================================================================

\section{Introduction}

\glnote{Distinction between statistical and systematic uncertainty.}
\glnote{Define nuisance parameters.}
\glnote{We want to build an accurate classifier whose output remains invariant with
respect to systematic uncertainties.}
% https://www.slac.stanford.edu/econf/C030908/papers/TUAT004.pdf
\glnote{Motivate the criterion (which may not be obvious for the ML crowd). See pivotal quantity motivation.}

% ==============================================================================


\section{Problem statement}
\label{sec:problem}

Let assume a probability space $(\Omega, {\cal F}, P)$, where $\Omega$ is a
sample space, ${\cal F}$ is a set of events and $P$ is a probability measure.
Let consider the multivariate random variables $X_z: \Omega \mapsto
\mathbb{R}^p$ and $Y: \Omega \mapsto {\cal Y}$, where $X_z$ denotes a dependence on a
nuisance parameter $Z$ whose values $z \in {\cal Z}$  define a
parameterized family of its systematic uncertainties. That is, $X_z$ and
$Y$ induce together a joint probability distribution $p(X,Y|z)$, where the
conditional on $z$ denotes $X_z$. For training, let further assume a
finite set $\{ x_i, y_i, z_i \}_{i=1}^N$ of realizations
$X_{z_i}(\omega_i), Y(\omega_i)$, for $\omega_i \in \Omega$ and known
values $z_i$ of the nuisance parameter. Our goal is to learn a function
$f(\cdot;\theta_f) : \mathbb{R}^p \mapsto {\cal Y}$ of parameters $\theta_f$
(e.g., a neural network-based classifier if ${\cal Y}$ is a finite set of
classes) and minimizing  a loss ${\cal L}_f(\theta_f)$ (e.g., the cross-entropy). In addition, we require
that $f(X_z ; \theta_f)$ should be robust to the value $z$ of the nuisance parameter  --
which remains unknown at test time. More specifically, we aim at building $f$
such that in the ideal case
\begin{equation}\label{eqn:criterion-true}
f(X_{z}(\omega) ; \theta_f) = f(X_{z^\prime}(\omega) ; \theta_f)
\end{equation} for all
samples $\omega \in \Omega$ and all $z, z^\prime$ pairs of values of the
nuisance parameter.

Since we do not have training tuples $(X_{z}(\omega),
X_{z^\prime}(\omega))$ (for the same unknown $\omega$), we propose instead to
solve the closely related problem of finding a predictive function $f$ such that
\begin{equation}\label{eqn:criterion-measure}
    P(\{ \omega | f(X_{z}(\omega) ; \theta_f) = y \} ) = P( \{ \omega' | f(X_{z^\prime}(\omega') ; \theta_f) = y\}) \text{ for all $y \in {\cal Y}$}.
\end{equation}
In words, we are looking for a predictive function $f$ which is a pivotal
quantity \citep{degroot1986probability} with respect to the nuisance parameter.
That is, such that  the distribution of $f(X_z; \theta_f)$ is invariant
with respect to the value $z$ of the nuisance. Note that a function $f$ for which
Eqn.~\ref{eqn:criterion-true} is true necessarily satisfies
Eqn.~\ref{eqn:criterion-measure}. In general, the converse is however not true, since the
sets of samples $\{ \omega | f(X_{z}(\omega); \theta_f) = y \}$ and $\{
\omega' | f(X_{z^\prime}(\omega'); \theta_f) = y \}$ do not need to be the same
for the equality to hold.
In order to simplify notations,
and as only Eqn.~\ref{eqn:criterion-measure} is
of direct interest in this work, we denote from here on
the pivotal quantity criterion as
\begin{equation}\label{eqn:criterion}
    p(f(X ; \theta_f) | z ) = p(f(X ; \theta_f) | z^\prime ) \text{ for all $z,z^\prime \in  {\cal Z}$.}
\end{equation}


% ==============================================================================

\section{Method}
\label{sec:method}

Adversarial training was first proposed by \cite{goodfellow2014generative} as a
way to build a generative model capable of producing samples from random noise
$z \sim p_Z$. More specifically, the authors pit a generative model $g: \mathbb{R}
\mapsto \mathbb{R}^p$ against an adversary classifier $d : \mathbb{R}^p \mapsto \{
0, 1\}$ whose antagonistic objective is to recognize real data $X$ from generated data $g(Z)$. Both
models $g$ and $d$ are trained simultaneously, in such a way that $g$ learns to
produce samples that are difficult to identify by $d$, while $d$ incrementally
adapts to changes in $g$. At the equilibrium, $g$ models a distribution whose
samples can be identified by $d$ only by chance. That is, assuming enough
capacity in $d$ and  $g$, the distribution $p_{g(Z)}$ eventually converges
towards the real distribution $p_X$.

In this work, we repurpose adversarial training as a means to constraint the
predictive model $f$ in order to satisfy Eqn.~\ref{eqn:criterion}. In
particular, we pit $f$ against an adversary model $r := p_{\theta_r}(z |
f(X;\theta_f))$ of parameters $\theta_r$ and associated loss ${\cal
L}_r(\theta_f, \theta_r)$. This model takes  as input realizations of $f(X;
\theta_f)$, for the current value $\theta_f$ of $f$ parameters, and produces as
output probability estimates $p_{\theta_r}(z | f(X;\theta_f))$ that $f(X;
\theta_f)$ is generated from the nuisance value $z$. Intuitively, if $p(f(X; \theta_f)|z)$
varies with $z$, then the corresponding correlation can be captured by $r$. By
contrast, if $p(f(X; \theta_f)|z)$ is invariant with $z$, as we require, then
$r$ should perform poorly and be close to random guessing. Training $f$ such
that it additionally minimizes the performance of $r$ therefore acts as a
regularization towards Eqn.~\ref{eqn:criterion}.

As for generative adversarial networks, we propose to
train $f$ and $r$ simultaneously, which we carry out by considering
the value function
\begin{equation}
    E(\theta_f, \theta_r) = {\cal L}_f(\theta_f) - {\cal L}_r(\theta_f, \theta_r)
\end{equation}
that we optimize by finding the saddle point $(\smash{\hat\theta_f}, \smash{\hat\theta_r})$ such that
\begin{align}
    \smash{\hat\theta_f} &= \arg \min_{\theta_f} E(\theta_f, \smash{\hat\theta_r}), \label{eqn:min_thetaf} \\
    \smash{\hat\theta_r} &= \arg \max_{\theta_r} E(\smash{\hat\theta_f}, \theta_r) \label{eqn:max_thetar}.
\end{align}
Without loss of generality, the adversarial training procedure to obtain
$(\smash{\hat\theta_f}, \smash{\hat\theta_r})$ is formally presented in
Algorithm~\ref{alg:adversarial-training} in the case of a binary classifier $f :
\mathbb{R}^p \mapsto [0,1]$ modeling $p(Y=1|X)$. For reasons further explained
in Section~\ref{sec:theory}, ${\cal L}_f$ (resp. ${\cal L}_r$) is set to the
expected value of the
negative log-likelihood of $Y|X$ under $f$ (resp. of $Z|f(X;\theta_f)$ under
$r$). The optimization algorithm consists in using stochastic gradient descent
alternatively for solving Eqn.~\ref{eqn:min_thetaf} and \ref{eqn:max_thetar}.

\begin{algorithm}[t]
\caption{Adversarial training of a classifier $f$ against an adversary $r$.\\
{\it Inputs:} training data $\{ x_i, y_i, z_i \}_{i=1}^N$\\
{\it Outputs:} $\smash{\hat\theta_f}, \smash{\hat\theta_r}$\\
{\it Hyper-parameters:} Number $T$ of training iterations, Number $K$ of gradient steps to update $r$.}
\label{alg:adversarial-training}
\begin{algorithmic}[1]
    \For{$t=1$ to $T$}
        \For{$k=1$ to $K$} \Comment{Update $r$}
            \State{Sample minibatch $\{x_m, z_m \}_{m=1}^M$} of size $M$;
            \State{With $\theta_f$ fixed, update $r$ by ascending its stochastic gradient $\nabla_{\theta_r} E(\theta_f, \theta_r) :=$
            $$\nabla_{\theta_r} \sum_{m=1}^M \log p_{\theta_r}(z_m|f(x_m;\theta_f))  ;$$}
        \EndFor
        \State{Sample minibatch $\{x_m, y_m, z_m \}_{m=1}^M$} of size $M$; \Comment{Update $f$}
        \State{With $\theta_r$ fixed, update $f$ by descending its stochastic gradient $\nabla_{\theta_f} E(\theta_f, \theta_r) :=$
        $$\nabla_{\theta_f}  \sum_{m=1}^M \left[ -\log p_{\theta_f}(y_m|x_m)  +\log p_{\theta_r}(z_m|f(x_m;\theta_f))  \right],$$
        \indent where $p_{\theta_f}(y_m|x_m)$ denotes $1(y_m=0)(1-f(x_m;\theta_f)) + 1(y_m=1)f(x_m;\theta_f))$;}
    \EndFor
\end{algorithmic}
\end{algorithm}


% ==============================================================================

\section{Theoretical results}
\label{sec:theory}

In this section, we show that in the setting of
Algorithm~\ref{alg:adversarial-training} where ${\cal L}_f$ and ${\cal L}_r$ are
respectively set to expected value of the negative log-likelihood of $Y|X$ under
$f$ and of $Z|f(X;\theta_f)$ under $r$, the procedure converges to a classifier
$f$ which is a pivotal quantity in the sense of Eqn.~\ref{eqn:criterion}.

In this setting, the nuisance parameter $Z$ is considered as a random variable,
for which we require the uniform prior $p(z)$ (for $z$ in ${\cal Z}$).
Importantly, classification of $Y$ with respect to $X$ is therefore considered
in the context where $Z$ is marginalized out, which means that the classifier
minimizing ${\cal L}_f$ is optimal with respect to $Y|X$, but not necessarily
with $Y|X,Z$. Results
hold for a nuisance parameter $Z$ taking either categorical values or continuous
values within a bounded support. By abuse of notation, $H(p_Z)$ denotes the
differential entropy in this latter case. Finally, propositions below are derived in a
non-parametric setting, by assuming that both $f$ and $r$ have enough capacity.

\begin{proposition}\label{prop:1}
Let $\theta_f$ be fixed and $\hat\theta_r = \arg \max_{\theta_r} E(\theta_f,
\theta_r)$. If $p_{\theta_r}(z|f(X;\theta_f)) = p(z)$
for all $z \in {\cal Z}$, then $f$ is a pivotal quantity.
\end{proposition}

\begin{proof}
The optimal parameters $\smash{\hat\theta}_r =
\arg \max_{\theta_r} E(\theta_f, \theta_r) = \arg \min_{\theta_r} {\cal L}_r(\theta_f,
\theta_r)$ are such that $p_{\theta_r}(z|f(X;\theta_f)) = p(z|f(X;\theta_f))$.
By assumption, $p_{\theta_r}(z|f(X;\theta_f)) = p(z)$,
and therefore $p(z|f(X;\theta_f)) = p(z)$.
Using the Bayes' rule, we write
\begin{align*}
    p(f(X;\theta_f)|z) &= \frac{ p(z|f(X;\theta_f)) p(f(X;\theta_f)) } { p(z)} = p(f(X;\theta_f)),
\end{align*}
which holds for all $z \in {\cal Z}$ and implies that $f$ is a pivotal quantity.
\end{proof}

\begin{proposition}\label{prop:2}
If there exists a saddle point $(\smash{\hat\theta}_f, \smash{\hat\theta}_r)$
for Eqn.~\ref{eqn:min_thetaf} and \ref{eqn:max_thetar} such that
$E(\hat\theta_f, \hat\theta_r) = H(p_{Y|X}) - H(p_Z)$, then
$f(\cdot;\smash{\hat\theta}_f)$ is both an optimal classifier  and a pivotal quantity.
\end{proposition}

\begin{proof}
For fixed $\theta_f$, the adversary $r$ is optimal at $\hat\theta_r = \arg
\max_{\theta_r} E(\theta_f, \theta_r)  = \arg \min_{\theta_r} {\cal L}_r(\theta_f,
\theta_r)$, in which case $p_{\theta_r}(z|f(X;\theta_f)) = p(z|f(X;\theta_f))$ and ${\cal L}_r$ reduces to the entropy
$H(p_{Z|f(X;\theta_f)})$ of the conditional distribution of the nuisance. The
value function $E$ can therefore be rewritten as $$E'(\theta_f) = {\cal L}_f(\theta_f) -
H(p_{Z|f(X;\theta_f)}).$$  In particular, we have the lower bound $H(p_{Y|X}) -
H(p_Z) \leq {\cal L}_f(\theta_f) - H(p_{Z|f(X;\theta_f)})$ where the equality
holds at $\smash{\hat\theta}_f = \arg \min_{\theta_f} E'(\theta_f)$ only when
\begin{itemize}
    \item $\smash{\hat\theta}_f$ corresponds to
    the parameters of an optimal classifier, in which case the expected negative log-likelihood ${\cal L}_f$ of $Y|X$
    reduces to its minimum value $H(p_{Y|X})$,
    \item all
   outcomes of $Z|f(X;\smash{\hat\theta}_f)$ are equally likely, in which case
   $p(z|f(X;\smash{\hat\theta}_f)) = p(z)$ for all $z \in {\cal
   Z}$ since we require a uniform prior by construction. Note that in the continuous case, the supremum of the differential entropy
   over continuous distributions on the same bounded support is also realized by the uniform distribution over that support.
\end{itemize}
Accordingly, the second condition implies that $p_{\theta_r}(z|f(X;\theta_f)) = p(z)$ and therefore that at this
point, because of Proposition~\ref{prop:1}, the optimal classifier $f(\cdot;\smash{\hat\theta}_f)$ is also a pivotal quantity.
\end{proof}

Proposition~\ref{prop:2} suggests that if at each step of
Algorithm~\ref{alg:adversarial-training} the adversary $r$ is allowed to reach
its optimum given $f$ (e.g., by setting $K$ sufficiently high) and if $f$ is
updated to improve ${\cal L}_f(\theta_f) - H(p_{Z|f(X;\theta_f)})$, then $f$
should converge to a classifier which is both optimal and pivotal, provided such
a classifier exists. On many problems of interest though, such a classifier may
not exist because the nuisance parameter directly shapes the decision
boundary, in which cases the lower bound $H(p_{Y|X}) - H(p_Z) < {\cal
L}_f(\theta_f) - H(p_{Z|f(X;\theta_f)})$ is strict: $f$ can either be an optimal
classifier or a pivotal quantity, but not both simultaneously. Accordingly the
value function $E$ can be rewritten as
\begin{equation}
    E_\lambda(\theta_f, \theta_r) = {\cal L}_f(\theta_f) - \lambda {\cal L}_r(\theta_f, \theta_r),
\end{equation}
where $\lambda \geq 0$ is a hyper-parameter controlling the trade-off between the performance of $f$
and its independence with respect to the nuisance parameter. Setting $\lambda$ to a large
value will preferrably enforces $f$ to be pivotal while setting $\lambda$ close to $0$
will rather constraint $f$ to be optimal.


% ==============================================================================

\section{Experiments}

\glnote{Define the architecture of $r$ when $Z$ is categorical ($r$ is a
standard classifier) or continuous ($r$ is e.g a 2-output NN such that the
distribution of $Z|f(X)$ is modeled by a truncated gaussian of known support).}


% ==============================================================================

\section{Related work}

\glnote{Alternative to learning a parameterized classifier for LLR.}

\glnote{Similar to domain adaptation, but with infinitely many domains,
as parameterized by $Z$, also related to transfer learning.}

\glnote{Other applications: removing implicit bias in data (e.g. gender bias).}


% ==============================================================================

\section{Conclusions}


% ==============================================================================

\section*{Acknowledgments}

\bibliographystyle{ieeetr}
\bibliography{bibliography.bib}

\end{document}
